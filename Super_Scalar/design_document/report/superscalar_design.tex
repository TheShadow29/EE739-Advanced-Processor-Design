\documentclass{article}
\usepackage[a4paper, tmargin=1in, bmargin=1in]{geometry}
\usepackage[utf8]{inputenc}
\usepackage{graphicx}
\usepackage[justification=centering]{caption}

% \usepackage{parskip}
\usepackage{pdflscape}
\usepackage{listings}
\usepackage{hyperref}
\usepackage{caption}
\usepackage{subcaption}
\usepackage{float}

\title{EE739A - Advanced Processor Design\\
  Project I : Superscalar IITB RISC
}
\author{Meet Udeshi - 14D070007\\
  OV Shashank - 14D070021\\
  Arka Sadhu - 140070011\\
}
\date{\today}

\begin{document}
\maketitle

\section{Fetch}
\subsection{Program Counter}
\begin{itemize}
\item This register is the speculative program counter which is used to fetch from the instruction cache
\item The actual permanent PC is stored in the ARF corresponding to register R7 as R7 == PC in the ISA
\item It is updated by PC+2 every cycle, unless a branch misprediction correction is issued or the fetch stage is stalled
\end{itemize}
\subsection{Branch Predictor}
\begin{itemize}
\item Stores Target Address and Prediction History corresponding to BEQ and JAL
\item JLR is not stores becuase its target address needs to always be computed and can be multitargeted which adds issues in checking whether the branch taken was to the right address or not
\item Branches are by default assumed to be not taken so that PC+2 rule can be continually followed
\item Planned to be implemented as a two bit predictor because the ISA does not allow from good prediction accuracy with small hardware
\end{itemize}

\section{Decode}
\subsection{Decoders}
\begin{itemize}
\item There are two paraller decoders which generate the necessary control signals for the future stages. This also includes the necessary reordering of the operands into source and destination registers
\item They include necessary hardware to detect inter-dependencies between the two instructions fetched
\item They generate inter-dependency bits which are later to be used in the renaming stage by the ARF and RRF to correctly generate and provide the tags
\item The validity of the destination operand for the ARF as well as for the carry and zero flags are produced which is passed on as the invalid tag bits in the execution stage
\end{itemize}
\subsection{LM/SM Handling Block}
\begin{itemize}
\item It is responsible for replacing the decode block in case of arrival of an LM or SM
\item It generates the necessary operands and addresses that are to be written to or read from. This is performed by generating signals similar to the load instruction and simply by incrementing the immediate field
\item To ensure that the value read from Address Source Register does not change an isLM bit is generated which is a signal for harware in the dispatch stange which ensures correct execution of the instruction
\end{itemize}

\section{Dispatch}
\subsection{Register Renaming}
Registers are 16-bit.
Carry and Zero Flag registers are 1-bit.
Renaming is performed for both the registers individually.
\subsubsection{Architectural Register File}
\begin{itemize}
\item They store tags of the corresponding rename regiser for every Architectural registers, along with non-speculative data.
\item It will also have a valid bit. Whenver interrupt all valid bits are set to 1.
\item If the AR is renamed, then it is invalid, else it is valid.
\item It is set valid, only when the tag pointed by it, and the tag broadcasted by the ROB matches.
\item It will take into account the interdependency bits and use the RRF queue to decide which tags to issue and which to be provided to the 
reservation station in case of a intra RAW hazard.
\end{itemize}
\subsubsection{Rename Register File}
\begin{itemize}
\item Execution will give the tag for the RR, and the corresponding register will update its value.
\item This will also be broadcasted RS.
\end{itemize}
\subsubsection{RRF Queue}
\begin{itemize}
\item Queue of available rename registers, updated based on ROB tag broadcast.
\end{itemize}
\subsection{Reservation Station}
\subsubsection{Allocation Policy}
\begin{itemize}
\item Keep a queue (circular), to store the available RS entries for allocation.
\item This has to check whether or not to stall.
\item The register values are from the pipeline registers, valid bits from the ARF, and ROB tag from ROB.
\item It also allocates to the ROB, and checks for stalls.
\end{itemize}
\subsubsection{Issue Policy}
\begin{itemize}
\item Top-down search for ready bits.
\item Decides between alu, branch, memory, as well as memory branch Execution pipes.
\end{itemize}
\subsubsection{Table}
\begin{itemize}
\item Source Rename Register Tags.
\item Source Register Values and Immediate Data
\item ROB tag
\item Control Signals including invalidTag bits
\end{itemize}
\subsubsection{LM/SM Handling Additions}
\begin{itemize}
\item Registers are necessary to store the tag as well as the value of the starting address of the LM/SM instruction
\item Storage of tag is necessary if the Address generating instruction (AGI) has not yet cleared the busy bit. This allows for the read to occur when the address is generated by the execution units
\item Storing the value is necessary because, once the AGI exits the ROB, the register tag stored would enter the RRF Queue and the value may be replaced which would lead to the remaining LM instructions reading the wrong instructions. And hence once tha value has been written to the RRF, it is simultaneously written to this value store register (VSR) and a valid bit alongside is set to say that the value in the VSR is valid and that the tag must not be read anymore.
\end{itemize}

\section{Execution}
\subsection{4 Pipes}

\section{Write Back}
\subsection{Reorder Buffer (ROB)}

\end{document}
